% abtex2-modelo-artigo.tex, v-1.9.2 laurocesar
% Copyright 2012-2014 by abnTeX2 group at http://abntex2.googlecode.com/ 
%

% ------------------------------------------------------------------------
% ------------------------------------------------------------------------
% abnTeX2: Modelo de Artigo Acadêmico em conformidade com
% ABNT NBR 6022:2003: Informação e documentação - Artigo em publicação 
% periódica científica impressa - Apresentação
% ------------------------------------------------------------------------
% ------------------------------------------------------------------------

\documentclass[
	% -- opções da classe memoir --
	article,			% indica que é um artigo acadêmico
	11pt,				% tamanho da fonte
	oneside,			% para impressão apenas no verso. Oposto a twoside
	a4paper,			% tamanho do papel. 
	% -- opções da classe abntex2 --
	%chapter=TITLE,		% títulos de capítulos convertidos em letras maiúsculas
	%section=TITLE,		% títulos de seções convertidos em letras maiúsculas
	%subsection=TITLE,	% títulos de subseções convertidos em letras maiúsculas
	%subsubsection=TITLE % títulos de subsubseções convertidos em letras maiúsculas
	% -- opções do pacote babel --
	portuguese,			% idioma adicional para hifenização
	brazil,				% o último idioma é o principal do documento
	sumario=tradicional
	]{abntex2}


% ---
% PACOTES
% ---

% ---
% Pacotes fundamentais 
% ---
\usepackage{lmodern}			% Usa a fonte Latin Modern
\usepackage[T1]{fontenc}		% Selecao de codigos de fonte.
\usepackage[utf8]{inputenc}		% Codificacao do documento (conversão automática dos acentos)
\usepackage{indentfirst}		% Indenta o primeiro parágrafo de cada seção.
\usepackage{nomencl} 			% Lista de simbolos
\usepackage{color}				% Controle das cores
\usepackage{graphicx}			% Inclusão de gráficos
\usepackage{microtype} 			% para melhorias de justificação
\usepackage{array}
\usepackage{tabularx}
\usepackage{graphicx}
\usepackage{subcaption}
\usepackage{float}
\usepackage{hyperref}
\usepackage{multirow}
\usepackage{longtable}
\usepackage{dpfloat, booktabs}
% ---
\graphicspath{ {images/} }		
% ---
% Pacotes adicionais, usados apenas no âmbito do Modelo Canônico do abnteX2
% ---
\usepackage{lipsum}				% para geração de dummy text
% ---
		
% ---
% Pacotes de citações
% ---
\usepackage[brazilian,hyperpageref]{backref}	 % Paginas com as citações na bibl
\usepackage[alf]{abntex2cite}	% Citações padrão ABNT
% ---

% ---
% Configurações do pacote backref
% Usado sem a opção hyperpageref de backref
\renewcommand{\backrefpagesname}{Citado na(s) página(s):~}
% Texto padrão antes do número das páginas
\renewcommand{\backref}{}
% Define os textos da citação
\renewcommand*{\backrefalt}[4]{
	\ifcase #1 %
		Nenhuma citação no texto.%
	\or
		Citado na página #2.%
	\else
		Citado #1 vezes nas páginas #2.%
	\fi}%
% ---

% ---
% Informações de dados para CAPA e FOLHA DE ROSTO
% ---
\titulo{Introdução a Compiladores - Trabalho 2 }
\autor{Daniel Yoshizawa (13101269)\\Guilherme Nunes (13103611)\\Larissa Taw (14209793)\\Mayse Espíndola (11204360) }
\local{Florianópolis, SC,  Brasil}
\data{25 de abril de 2018}
% ---

% ---
% Configurações de aparência do PDF final

% alterando o aspecto da cor azul
\definecolor{blue}{RGB}{41,5,195}

% informações do PDF
\makeatletter
\hypersetup{
     	%pagebackref=true,
		pdftitle={\@title}, 
		pdfauthor={\@author},
    	pdfsubject={Modelo de artigo científico com abnTeX2},
	    pdfcreator={LaTeX with abnTeX2},
		pdfkeywords={abnt}{latex}{abntex}{abntex2}{atigo científico}, 
		colorlinks=true,       		% false: boxed links; true: colored links
    	linkcolor=blue,          	% color of internal links
    	citecolor=blue,        		% color of links to bibliography
    	filecolor=magenta,      		% color of file links
		urlcolor=blue,
		bookmarksdepth=4
}
\makeatother
% --- 

% ---
% compila o indice
% ---
\makeindex
% ---

% ---
% Altera as margens padrões
% ---
\setlrmarginsandblock{3cm}{3cm}{*}
\setulmarginsandblock{3cm}{3cm}{*}
\checkandfixthelayout
% ---

% --- 
% Espaçamentos entre linhas e parágrafos 
% --- 

% O tamanho do parágrafo é dado por:
\setlength{\parindent}{1.3cm}

\setlength{\parskip}{0.2cm}

\SingleSpacing

\begin{document}

\frenchspacing 

\maketitle

\begin{resumoumacoluna}
Criação de um analisador sintático utilizando a ferramenta de geração de código Javacc para a disciplina de Introdução a compiladores, neste documento
são apresentadas as dificuldades encontradas para a geração do analisador sintático assim como a tabela de \textit{outputs} para os arquivos de código fonte analisados e a explicação
da solução dos items A,B,C e D do exercício 3.

 \vspace{\onelineskip}
 
 \noindent
 \textbf{Palavras-chaves}: Introdução a Compiladores, Javacc, Fun, Analisador Sintático
\end{resumoumacoluna}

\newpage
\tableofcontents*
\newpage

\textual
\section{Descrição das atividade}

Foi construído um analisador sintático para a linguagem de programação Fun, proposta na disciplina de Introdução a compiladores, baseada em Java e utilizando a 
ferramenta Javacc para geração de código.

Nesse trabalho foram definidas as regras de analise sintática seguindo as sessões 2.3, 2,4 e 2,5 do livro \textit{Como Construir um Compilador Utilizando Ferramentas Java}, assim como a criação de arquivos fontes para programas compilados na linguagem que estamos desenvolvendo, com a finalidade de testar a integridade e funcionalidades inseridas no compilador, foram feitos programas que devem ser aceitos e que não devem ser aceitos pelo compilador.
 

\section{Dificuldades encontradas}

Durante o desenvolvimento deste trabalho encontramos algumas dificuldades que foram logo superadas com o material de apoio e trabalho em grupo, entre elas determinar as expressões que seriam avaliadas para aceitar os operadores lógicos, tanto em relação a utilização de parênteses como ordem de precedência, porem após analise optamos pela solução apresentada no trabalho.

Como já possuímos alguma familiaridade com a ferramenta \textbf{JavaCC} e o conteúdo deste trabalho é de maior conhecimento da equipe, a execução dessa atividade foi um pouco mais fácil que ao trabalho anterior portanto não temos muitas dificuldades a apresentar.

\section{Soluções aplicadas}

Para definir as regras da linguagem nos baseamos em linguagens de programação já consagradas, como Java e C++, seguindo algumas das regras utilizadas por estas
linguagens, assim como soluções que foram discutidas em grupo para chegar a uma solução interessante para o fim deste trabalho.

Com o auxilio do livro base e da documentação da ferramenta foi possível  sanar a maioria das nossas duvidas em relação ao Javacc e com o material e explicações das aulas
pudemos criar as expressões regulares que julgamos mais adequadas a solução do problema.

\section{Solução do item 3}

\subsection{3.a}

Criamos os novos tokens dos novos tipos, \textbf{float}, \textbf{boolean} e \textbf{char}, então verificamos onde cada outro tipo era utilizado em declarações e expressões e acrescentamos no \textbf{REGEX} as opções de novos tokens.

\subsection{3.b}

Foi verificado onde era executado a soma, divisão etc, visto que os métodos se chamavam na ordem de precedência foi acrescentado os novos operadores lógicos, not, and, or e xor,  com a ordem de acordo com o enunciado, depois fizemos o not ficar mais a esquerda.

\subsection{3.c}

Verificamos onde as variáveis eram inicializadas, função \textit{vardecl()} e acrescentamos o <ASSIGN>, sendo que toda vez que tiver um <ASSIGN> obrigatoriamente tem que ter uma expressão depois.

Depois foi acrescido a verificação na lista de parâmetros e dentro de métodos.

\subsection{3.d}

Havia na ordem \textit{x} ou mais construtores e depois \textit{x} ou mais métodos, alteramos pra ter \textit{x} ou mais métodos ou construtores, podendo assim ser em qualquer ordem.

\newpage
\section{Tabela de tokens}

Abaixo são apresentadas as tabelas com os programs testados, aceitos e negados assim como o output gerado pelo analisador sintático.

\subsection{Aceitos}

\begin{center}
\begin{table}[H]
\begin{tabularx}{1\textwidth}{p{5cm}|X}
Arquivo & Output \\
\hline
"programaaceito1.fun" & "O programa foi corretamente analisado" \\
"programaaceito2.fun" & "O programa foi corretamente analisado" \\
"programaaceito3.fun" & "O programa foi corretamente analisado" \\
"programaaceito4.fun" & "O programa foi corretamente analisado" \\ \hline
\end{tabularx}
\end{table}
\end{center}

\subsection{Negados}

\begin{center}
\begin{table}[H]
\begin{tabularx}{1\textwidth}{p{5cm}|X}
Arquivo & Output \\
\hline
programanaoaceito1.fun & Encountered '' ''class'' ''class '''' at line 1, column 35. Was expecting: <IDENT> ... \\
programanaoaceito2.fun & Encountered '' ''boolean'' ''boolean '''' at line 3, column 21. Was expecting:''('' ... \\
programanaoaceito3.fun &  Encountered '' ''int'' ''int '''' at line 4, column 6. Was expecting one of: '']'' ... <IDENT> ... \\
programanaoaceito4.fun &  Encountered '' ''='' ''= '''' at line 5, column 9. Was expecting one of: '']'' ... <IDENT> ...\\
programanaoaceito5.fun &  Encountered '' '')'' '') '''' at line 5, column 21. Was expecting one of: <IDENT> ...\\
programanaoaceito6.fun &  Encountered '' <NOT> ''not '''' at line 5, column 18. Was expecting one of: \
    ''+'' ... \
    ''-'' ... \
    ''*'' ... \
    ''/'' ... \
    <AND> ... \
    <OR> ... \ 
    <XOR> ... \ 
    ''>'' ... \ 
    ''<'' ... \
    ''=='' ... \ 
    ''<='' ... \ 
    ''>='' ... \ 
    <NEQ> ...\ 
    ''\%'' ... \
    '')'' ...\\
programanaoaceito7.fun & Encountered '' '')'' '') '''' at line 5, column 31. Was expecting one of: \
    ''+'' ... \
    ''-'' ... \
    <NOT> ... \
    ''('' ... \
    <float\_constant> ... \
    <integer\_constant> ... \
    <boolean\_constant> ... \
    <char\_constant> ... \
    <string\_constant> ... \
    ''null'' ... \
    <IDENT> ... \\
programanaoaceito8.fun & Encountered " <AND> "and "" at line 5, column 37. Was expecting one of:\
    "+" ...\
    "-" ...\
    <NOT> ...\
    "(" ...\
    <float\_constant> ...\
    <integer\_constant> ...\
    <boolean\_constant> ...\
    <char\_constant> ...\
    <string\_constant> ...\
    "null" ...\
    <IDENT> ... \\
programanaoaceito9.fun & Encountered " ")" ") "" at line 5, column 21. Was expecting one of:\
    "+" ...\
    "-" ...\
    <NOT> ...\
    "(" ...\
    <float\_constant> ...\
    <integer\_constant> ...\
    <boolean\_constant> ...\
    <char\_constant> ...\
    <string\_constant> ...\
    "null" ...\
    <IDENT> ...\\
programanaoaceito10.fun & Encountered " <NOT> "not "" at line 5, column 19. Was expecting one of:\
    "+" ... \
    "-" ... \
    "*" ... \
    "/" ...\
    <AND> ...\
    <OR> ...\
    <XOR> ...\
    ">" ...\
    "<" ...\
    "==" ...\
    "<=" ...\
    ">=" ...\
    <NEQ> ...\
    "\%" ...\
    "float" ...\
    "boolean" ...\
    "char" ...\
    "string" ...\
    "int" ...\
    "break" ...\
    "print" ...\
    "read" ...\
    "return" ...\
    "super" ...\
    "if" ...\
    "for" ...\
    "\{" ...\
    "\}" ...\
    ";" ...\
    <IDENT> ... \\
programanaoaceito11.fun & Encountered '' ''='' ''= '''' at line 3, column 26. Was expecting one of: <IDENT> ...  \\
programanaoaceito12.fun & Encountered " <string\_constant> "\"aaaa\" "" at line 3, column 37. Was expecting one of:\
    "=" ...\
    ")" ...\
    "]" ...\
    "," ...\\ \hline
\end{tabularx}
\end{table}
\end{center}

\newpage
\section{Conclusão}

Com esse trabalhos conseguimos compreender melhor o funcionamento de um analisador sintático assim como o entendimento do uso da ferramenta Javacc para criação de um compilador.

Com o presente concluímos a segunda etapa da série de trabalhos da disciplina com o intuito final de construir todas as etapas de um compilador, gerando assim uma linguagem de programação que possa ser compilada utilizando o material gerado nestes trabalhos.

\bookmarksetup{startatroot}
\end{document}
