% abtex2-modelo-artigo.tex, v-1.9.2 laurocesar
% Copyright 2012-2014 by abnTeX2 group at http://abntex2.googlecode.com/ 
%

% ------------------------------------------------------------------------
% ------------------------------------------------------------------------
% abnTeX2: Modelo de Artigo Acadêmico em conformidade com
% ABNT NBR 6022:2003: Informação e documentação - Artigo em publicação 
% periódica científica impressa - Apresentação
% ------------------------------------------------------------------------
% ------------------------------------------------------------------------

\documentclass[
	% -- opções da classe memoir --
	article,			% indica que é um artigo acadêmico
	11pt,				% tamanho da fonte
	oneside,			% para impressão apenas no verso. Oposto a twoside
	a4paper,			% tamanho do papel. 
	% -- opções da classe abntex2 --
	%chapter=TITLE,		% títulos de capítulos convertidos em letras maiúsculas
	%section=TITLE,		% títulos de seções convertidos em letras maiúsculas
	%subsection=TITLE,	% títulos de subseções convertidos em letras maiúsculas
	%subsubsection=TITLE % títulos de subsubseções convertidos em letras maiúsculas
	% -- opções do pacote babel --
	portuguese,			% idioma adicional para hifenização
	brazil,				% o último idioma é o principal do documento
	sumario=tradicional
	]{abntex2}


% ---
% PACOTES
% ---

% ---
% Pacotes fundamentais 
% ---
\usepackage{lmodern}			% Usa a fonte Latin Modern
\usepackage[T1]{fontenc}		% Selecao de codigos de fonte.
\usepackage[utf8]{inputenc}		% Codificacao do documento (conversão automática dos acentos)
\usepackage{indentfirst}		% Indenta o primeiro parágrafo de cada seção.
\usepackage{nomencl} 			% Lista de simbolos
\usepackage{color}				% Controle das cores
\usepackage{graphicx}			% Inclusão de gráficos
\usepackage{microtype} 			% para melhorias de justificação
\usepackage{array}
\usepackage{tabularx}
\usepackage{graphicx}
\usepackage{subcaption}
\usepackage{float}
\usepackage{hyperref}
\usepackage{multirow}
\usepackage{longtable}
\usepackage{dpfloat, booktabs}
% ---
\graphicspath{ {images/} }		
% ---
% Pacotes adicionais, usados apenas no âmbito do Modelo Canônico do abnteX2
% ---
\usepackage{lipsum}				% para geração de dummy text
% ---
		
% ---
% Pacotes de citações
% ---
\usepackage[brazilian,hyperpageref]{backref}	 % Paginas com as citações na bibl
\usepackage[alf]{abntex2cite}	% Citações padrão ABNT
% ---

% ---
% Configurações do pacote backref
% Usado sem a opção hyperpageref de backref
\renewcommand{\backrefpagesname}{Citado na(s) página(s):~}
% Texto padrão antes do número das páginas
\renewcommand{\backref}{}
% Define os textos da citação
\renewcommand*{\backrefalt}[4]{
	\ifcase #1 %
		Nenhuma citação no texto.%
	\or
		Citado na página #2.%
	\else
		Citado #1 vezes nas páginas #2.%
	\fi}%
% ---

% ---
% Informações de dados para CAPA e FOLHA DE ROSTO
% ---
\titulo{Introdução a Compiladores - Trabalho 4 }
\autor{Daniel Yoshizawa (13101269)\\Guilherme Nunes (13103611)\\Larissa Taw (14209793)\\Mayse Espíndola (11204360) }
\local{Florianópolis, SC,  Brasil}
\data{10 de junho de 2018}
% ---

% ---
% Configurações de aparência do PDF final

% alterando o aspecto da cor azul
\definecolor{blue}{RGB}{41,5,195}

% informações do PDF
\makeatletter
\hypersetup{
     	%pagebackref=true,
		pdftitle={\@title}, 
		pdfauthor={\@author},
    	pdfsubject={Modelo de artigo científico com abnTeX2},
	    pdfcreator={LaTeX with abnTeX2},
		pdfkeywords={abnt}{latex}{abntex}{abntex2}{atigo científico}, 
		colorlinks=true,       		% false: boxed links; true: colored links
    	linkcolor=blue,          	% color of internal links
    	citecolor=blue,        		% color of links to bibliography
    	filecolor=magenta,      		% color of file links
		urlcolor=blue,
		bookmarksdepth=4
}
\makeatother
% --- 

% ---
% compila o indice
% ---
\makeindex
% ---

% ---
% Altera as margens padrões
% ---
\setlrmarginsandblock{3cm}{3cm}{*}
\setulmarginsandblock{3cm}{3cm}{*}
\checkandfixthelayout
% ---

% --- 
% Espaçamentos entre linhas e parágrafos 
% --- 

% O tamanho do parágrafo é dado por:
\setlength{\parindent}{1.3cm}

\setlength{\parskip}{0.2cm}

\SingleSpacing

\begin{document}

\frenchspacing 

\maketitle

\begin{resumoumacoluna}

Criação da árvore sintática assim como o inicio da construção de um analisador semântico, contendo uma tabela de símbolos, utilizando a ferramenta de geração de código Javacc para a disciplina de Introdução a compiladores, neste documento são apresentadas as dificuldades encontradas para a solução do trabalho e as modificações necessárias para adaptar o programa aos trabalhos 2 e 3.

 \vspace{\onelineskip}
 
 \noindent
 \textbf{Palavras-chaves}: Introdução a Compiladores, Javacc, Fun, Analisador Semântico
\end{resumoumacoluna}

\newpage
\tableofcontents*
\newpage
\textual
\section{Descrição das atividade}

Foi construído uma árvore sintática e iniciada a criação do analisador semântico para a linguagem de programação Fun, proposta na disciplina de Introdução a compiladores, baseada em Java e utilizando a ferramenta Javacc para geração de código.

Neste trabalho foram criados os códigos para gerar a árvore sintática seguindo o capítulo 6 do livro base do trabalho assim como alterado o código, conforme capítulo 7, para exibir a árvore sintática após compilação do código base, não foi parametrizado no programa a opção de mostrar ou não a árvore impressa, decidimos por sempre exibir pois este se trata de um projeto didático e achamos que seria benéfico sempre imprimir a árvore. Foram adicionados outros tipos e operadores para se adequar as modificações oriundas dos trabalhos anteriores.

Também foi criada a tabela de símbolos com modificações, segundo o capítulo 8 do livro base, dentre as modificações está permitir que \textit{EntryRec} armazene se dado atributo possui valor \textit{default}.

A parte D sugere a implementação da primeira fase de análise semântica, que foi construída seguindo o capítulo 9 do livro, está sofreu algumas alterações para se adaptar aos trabalhos passados e foram adicionadas saídas de texto para informar quando uma classe era encontrada e quando está era adicionada a tabela de símbolos.
 
\section{Dificuldades encontradas}

Durante o desenvolvimento deste trabalho encontramos algumas dificuldades, entre elas o entendimento da organização dos arquivos, também a geração de um grande volume de novas classes e a geração de exemplos que abrangessem a maioria dos casos de erros possíveis e exemplos de funcionamento que fossem simples e claros.

O tempo gasto para desenvolver essa etapa do trabalho foi a maior até o momento e exigiu bastante leitura e comunicação do grupo para conseguirmos orquestrar a melhor maneira de concluir o trabalho.

\section{Soluções aplicadas}

O grupo manteve comunicação durante a maior parte do tempo na realização deste trabalho, nos ajudando a compreender e solucionar os problemas que foram encontrados, grande parte da solução foi leitura e geração de código similar ao apresentado no livro. Quando nos deparamos com situações em que eram necessárias alterações para compatibilidade de código nós discutimos e tentamos aplicar a solução que julgamos mais adequadas.

Um dos desafios foi qual o melhor lugar para imprimir que uma classe foi encontrada e quando ela é inserida na tabela de símbolos, porem com analise do código e consultas ao livro determinamos os dois pontos que para nos melhores se encaixam com o que foi pedido.

\newpage
\section{Conclusão}

Com esse trabalhos conseguimos compreender melhor o funcionamento do analisador sintático, tabela de símbolos e analisador semântico assim como o entendimento do uso da ferramenta Javacc para criação de um compilador.

Com o presente concluímos a quarta etapa da série de trabalhos da disciplina com o intuito final de construir todas as etapas de um compilador, gerando assim uma linguagem de programação que possa ser compilada utilizando o material gerado nestes trabalhos.

\bookmarksetup{startatroot}
\end{document}
