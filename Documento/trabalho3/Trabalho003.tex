% abtex2-modelo-artigo.tex, v-1.9.2 laurocesar
% Copyright 2012-2014 by abnTeX2 group at http://abntex2.googlecode.com/ 
%

% ------------------------------------------------------------------------
% ------------------------------------------------------------------------
% abnTeX2: Modelo de Artigo Acadêmico em conformidade com
% ABNT NBR 6022:2003: Informação e documentação - Artigo em publicação 
% periódica científica impressa - Apresentação
% ------------------------------------------------------------------------
% ------------------------------------------------------------------------

\documentclass[
	% -- opções da classe memoir --
	article,			% indica que é um artigo acadêmico
	11pt,				% tamanho da fonte
	oneside,			% para impressão apenas no verso. Oposto a twoside
	a4paper,			% tamanho do papel. 
	% -- opções da classe abntex2 --
	%chapter=TITLE,		% títulos de capítulos convertidos em letras maiúsculas
	%section=TITLE,		% títulos de seções convertidos em letras maiúsculas
	%subsection=TITLE,	% títulos de subseções convertidos em letras maiúsculas
	%subsubsection=TITLE % títulos de subsubseções convertidos em letras maiúsculas
	% -- opções do pacote babel --
	portuguese,			% idioma adicional para hifenização
	brazil,				% o último idioma é o principal do documento
	sumario=tradicional
	]{abntex2}


% ---
% PACOTES
% ---

% ---
% Pacotes fundamentais 
% ---
\usepackage{lmodern}			% Usa a fonte Latin Modern
\usepackage[T1]{fontenc}		% Selecao de codigos de fonte.
\usepackage[utf8]{inputenc}		% Codificacao do documento (conversão automática dos acentos)
\usepackage{indentfirst}		% Indenta o primeiro parágrafo de cada seção.
\usepackage{nomencl} 			% Lista de simbolos
\usepackage{color}				% Controle das cores
\usepackage{graphicx}			% Inclusão de gráficos
\usepackage{microtype} 			% para melhorias de justificação
\usepackage{array}
\usepackage{tabularx}
\usepackage{graphicx}
\usepackage{subcaption}
\usepackage{float}
\usepackage{hyperref}
\usepackage{multirow}
\usepackage{longtable}
\usepackage{dpfloat, booktabs}
% ---
\graphicspath{ {images/} }		
% ---
% Pacotes adicionais, usados apenas no âmbito do Modelo Canônico do abnteX2
% ---
\usepackage{lipsum}				% para geração de dummy text
% ---
		
% ---
% Pacotes de citações
% ---
\usepackage[brazilian,hyperpageref]{backref}	 % Paginas com as citações na bibl
\usepackage[alf]{abntex2cite}	% Citações padrão ABNT
% ---

% ---
% Configurações do pacote backref
% Usado sem a opção hyperpageref de backref
\renewcommand{\backrefpagesname}{Citado na(s) página(s):~}
% Texto padrão antes do número das páginas
\renewcommand{\backref}{}
% Define os textos da citação
\renewcommand*{\backrefalt}[4]{
	\ifcase #1 %
		Nenhuma citação no texto.%
	\or
		Citado na página #2.%
	\else
		Citado #1 vezes nas páginas #2.%
	\fi}%
% ---

% ---
% Informações de dados para CAPA e FOLHA DE ROSTO
% ---
\titulo{Introdução a Compiladores - Trabalho 2 }
\autor{Daniel Yoshizawa (13101269)\\Guilherme Nunes (13103611)\\Larissa Taw (14209793)\\Mayse Espíndola (11204360) }
\local{Florianópolis, SC,  Brasil}
\data{30 de maio de 2018}
% ---

% ---
% Configurações de aparência do PDF final

% alterando o aspecto da cor azul
\definecolor{blue}{RGB}{41,5,195}

% informações do PDF
\makeatletter
\hypersetup{
     	%pagebackref=true,
		pdftitle={\@title}, 
		pdfauthor={\@author},
    	pdfsubject={Modelo de artigo científico com abnTeX2},
	    pdfcreator={LaTeX with abnTeX2},
		pdfkeywords={abnt}{latex}{abntex}{abntex2}{atigo científico}, 
		colorlinks=true,       		% false: boxed links; true: colored links
    	linkcolor=blue,          	% color of internal links
    	citecolor=blue,        		% color of links to bibliography
    	filecolor=magenta,      		% color of file links
		urlcolor=blue,
		bookmarksdepth=4
}
\makeatother
% --- 

% ---
% compila o indice
% ---
\makeindex
% ---

% ---
% Altera as margens padrões
% ---
\setlrmarginsandblock{3cm}{3cm}{*}
\setulmarginsandblock{3cm}{3cm}{*}
\checkandfixthelayout
% ---

% --- 
% Espaçamentos entre linhas e parágrafos 
% --- 

% O tamanho do parágrafo é dado por:
\setlength{\parindent}{1.3cm}

\setlength{\parskip}{0.2cm}

\SingleSpacing

\begin{document}

\frenchspacing 

\maketitle

\begin{resumoumacoluna}
Criação de um analisador sintático utilizando a ferramenta de geração de código Javacc para a disciplina de Introdução a compiladores, neste documento
são apresentadas as dificuldades encontradas para a geração do analisador sintático assim como a tabela de \textit{outputs} para os arquivos de código fonte analisados.

 \vspace{\onelineskip}
 
 \noindent
 \textbf{Palavras-chaves}: Introdução a Compiladores, Javacc, Fun, Analisador Sintático
\end{resumoumacoluna}

\newpage
\tableofcontents*
\newpage
\textual

\section{Resumo seção 4.1}

Analisador Sintático recebe do analisador léxico uma cadeia de tokens, os quais serão validados para verificar se essa cadeia poderá ser gerada pela gramática da linguagem fonte.

Detectando erros de sintaxe nos tokens e a partir dos erros mais frequente ter a capacidade de recuperar-se deles, podendo assim continuar a análise do restante da entrada. 

Os três tipos de analisador sintáticos existem são: 

\begin{itemize}
	\item Métodos universais de análise sintática: com capacidade de tratar qualquer gramática
	\item Top-down: realiza a construção da árvore da raiz a para a folha. 
	\item Bottom-up: realiza a construção da árvore da folha até atingir a raiz. 
\end{itemize}

Os métodos top-down e bottom-up são considerados mais eficientes que os métodos universais de análise sintática, e seu processo utiliza apenas determinadas subclasses de gramáticas, como LL e LR, as quais já são consideradas suficientemente expressivas para construção sintática das linguagens de programação. 

Qualquer programa/código está passível a possuir erros de sintaxe, por mais atento e experiente que seja o programador que o desenvolveu, as linguagem de programação que deveriam auxiliar nessa validação de erro, não executam essa função, resto assim ao compilador executar a função de auxiliar de programador para localizar e identificar tais erros.  
	
A fase de análise sintática é responsável pela detecção e recuperação dos erros, devido a natureza do erro ser geralmente sintáticos e possuir um moderno e eficiente processo de detecção. Os erros detectados pelo analisador devem ser relatados de alguma forma, a mais comumente utilizada é a impressão da linha ilegal com um apontador para a posição na qual erro foi detectado. 

Referente a tratamento de erros o analisador sintático possui determinadas metas a serem cumpridas:

\begin{itemize}
	\item Deve relatar a presença de erros clara e acuradamente;
	\item Deve se recuperar de cada erro suficientemente rápido a fim de ser capaz de detectar erros subsequentes;
	\item Não deve retardar significativamente o processamento de programas corretos. 
\end{itemize}

Após o erro detectado o analisador deverá tentar realizar a recuperação do mesmo, pois encerrar o processo logo após encontrar o primeiro erro poderá resultar na não detecção dos possíveis erros que a mesma entrada pode conter. Porém deve-se cuidar para que a recuperação não acabe resultando nas gerações de novos erros, os quais não foram gerados pelos programadores. Existe várias estratégia distinta que um analisador sintático poderá utilizar para recuperar-se de um erro e mesmo não existindo uma escolha universal algumas estratégias são utilizadas mais frequentemente. Tais como: 

Recuperação na modalidade do desespero: é o mais simples de implementar e pode ser usado pela maioria dos métodos de análise sintática. Ao descobrir um erro, o analisador descarta símbolos de entrada, um de cada vez, até que seja encontrado um token pertencente a um conjunto designado de tokens de sincronização. Tal correção que frequentemente pula uma parte considerável da entrada sem verifica-la, procurando erros adicionais, possui vantagem da simplicidade e diferente dos demais métodos, tem a garantia de não entrar num laço infinito.  

Recuperação de frase: ao descobrir um erro, o analisador pode realizar uma correção local a entrada restante, substituído um prefixo da entrada remanescente por alguma cadeia que permita ao analisador seguir em frente. Naturalmente, deve-se ser cuidadoso nas escolhas das substituições para que não gere um loop infinito. A vantagem desse método é que essas  substituições pode corrigir qualquer cadeia porém possui a desvantagem de apresentar dificuldade em lidar com situações nas quais os erros efetivos ocorrem antes do ponto de detecção.  

Produções de erro: usa a gramática aumentada com produções de erros (erros comuns que ocorrem com frequência) para construir um analisador sintático. Gerando diagnóstico a partir das produções de erro utilizadas, para indicar as construções ilegais que foram reconhecidas na entrada. 

Correção global: a partir de uma cadeia incorreta de entrada e uma gramática o método irá encontrar uma árvore gramatical para uma cadeia relacionada, de tal forma que as mudanças realizadas sejam as menores possíveis. Devido ao seu alto custo de implementação e espaço, esse método é voltado apenas a interesses teóricos.

\section{Descrição das atividade}

Foi construído um analisador sintático para a linguagem de programação Fun, proposta na disciplina de Introdução a compiladores, baseada em Java e utilizando a 
ferramenta Javacc para geração de código.

Neste trabalho foi criado o tratamento de erros para o analisador sintático, seguindo o \textbf{método do pânico} descrito no livro base, o tratamento de erros tenta alcançar o maior numero de erros de uma única vez tentando evitar que o mesmo código precise ser recompilado varias vezes para encontrar diversos erros que possam existir.
 
\section{Dificuldades encontradas}

Durante o desenvolvimento deste trabalho encontramos algumas dificuldades que foram logo superadas com o material de apoio e trabalho em grupo, entre elas determinar as expressões que seriam avaliadas para aceitar os operadores lógicos, tanto em relação a utilização de parênteses como ordem de precedência, porem após analise optamos pela solução apresentada no trabalho.

Como já possuímos alguma familiaridade com a ferramenta \textbf{JavaCC} e o conteúdo deste trabalho é de maior conhecimento da equipe, a execução dessa atividade foi um pouco mais fácil que ao trabalho anterior portanto não temos muitas dificuldades a apresentar.

\section{Soluções aplicadas}

Para definir as regras da linguagem nos baseamos em linguagens de programação já consagradas, como Java e C++, seguindo algumas das regras utilizadas por estas
linguagens, assim como soluções que foram discutidas em grupo para chegar a uma solução interessante para o fim deste trabalho.

Com o auxilio do livro base e da documentação da ferramenta foi possível  sanar a maioria das nossas duvidas em relação ao Javacc e com o material e explicações das aulas
pudemos criar as expressões regulares que julgamos mais adequadas a solução do problema.

A solução adotada foi o \textbf{método do pânico}, onde as funções de analise são guardadas com a instrução \textbf{try catch} e é passado um objeto do tipo RecoverySet para fazer o track dos erros até o proximo símbolo terminal, onde retorna o erro encontrado.

\section{Outputs}
./exemplos/nao\_aceitos/programanaoaceito1.fun\\
Trabalho de Introdução a Compiladores.\\
Lendo o arquivo : ./exemplos/nao\_aceitos/programanaoaceito1.fun
Encountered " "class" "class "" at line 1, column 35.\\
Was expecting:\\
    <IDENT> ... \\
\\
O programa foi corretamente analisado\\
1 Erro sintatico encontrado\\
------\\
./exemplos/nao\_aceitos/programanaoaceito2.fun\\
Trabalho de Introdução a Compiladores.\\
Lendo o arquivo : ./exemplos/nao\_aceitos/programanaoaceito2.fun
Encountered " "boolean" "boolean "" at line 3, column 28.\\
Was expecting:\\
    "(" ...\\
\\
Encountered " ";" "; "" at line 3, column 35.
Was expecting one of:\\
    "]" ...\\
    <IDENT> ...\\
\\
O programa foi corretamente analisado\\
2 Erro sintatico encontrado\\
-------\\
./exemplos/nao\_aceitos/programanaoaceito3.fun\\
Trabalho de Introdução a Compiladores.\\
Lendo o arquivo : ./exemplos/nao\_aceitos/programanaoaceito3.fun\\
Encountered " "int" "int "" at line 4, column 13.\\
Was expecting one of:
    "]" ...\\
    <IDENT> ...\\
    <IDENT> ...\\
\\
O programa foi corretamente analisado\\
1 Erro sintatico encontrado\\
--------\\
./exemplos/nao\_aceitos/programanaoaceito4.fun\\
Trabalho de Introdução a Compiladores.\\
Lendo o arquivo : ./exemplos/nao\_aceitos/programanaoaceito4.fun\\
Encountered " "=" "= "" at line 5, column 16.\\
Was expecting one of:
    "]" ...\\
    <IDENT> ...\\
    <IDENT> ...\\
\\
O programa foi corretamente analisado\\
1 Erro sintatico encontrado\\
--------\\
./exemplos/nao\_aceitos/programanaoaceito5.fun\\
Trabalho de Introdução a Compiladores.\\
Lendo o arquivo : ./exemplos/nao\_aceitos/programanaoaceito5.fun
Encountered " ")" ") "" at line 5, column 28.\\
Was expecting:\\
    <IDENT> ...\\
\\
Encountered " "\}" "\} "" at line 7, column 9.\\
Was expecting one of:\\
    "break" ...\\
    "print" ...\\
    "read" ...\\
    "return" ...\\
    "super" ...\\
    "if" ...\\
    "for" ...\\
    "\{" ...
    ";" ...\\
    <IDENT> ...\\
    <IDENT> ...\\
\\
O programa foi corretamente analisado\\
2 Erro sintatico encontrado\\
--------\\
./exemplos/nao\_aceitos/programanaoaceito6.fun\\
Trabalho de Introdução a Compiladores.\\
Lendo o arquivo : ./exemplos/nao\_aceitos/programanaoaceito6.fun\\
Encountered " <NOT> "not "" at line 5, column 32.\\
Was expecting one of:\\
    "+" ...\\
    "-" ...\\
    "*" ...\\
    "/" ...\\
    <AND> ...\\
    <OR> ...\\
    <XOR> ...\\
    ">" ...\\
    "<" ...\\
    "==" ...\\
    "<=" ...\\
    ">=" ...
    <NEQ> ...\\
    "%" ...\\
    ")" ...\\
\\
O programa foi corretamente analisado\\
1 Erro sintatico encontrado\\
--------\\
./exemplos/nao\_aceitos/programanaoaceito7.fun\\
Trabalho de Introdução a Compiladores.\\
Lendo o arquivo : ./exemplos/nao\_aceitos/programanaoaceito7.fun\\
Encountered " ")" ") "" at line 5, column 45.\\
Was expecting one of:\\
    "+" ...\\
    "-" ...\\
    <NOT> ...\\
    "(" ...\\
    <float\_constant> ...\\
    <integer\_constant> ...\\
    <boolean\_constant> ...\\
    <char\_constant> ...
    <string\_constant> ...\\
    "null" ...\\
    <IDENT> ...\\
\\
O programa foi corretamente analisado\\
1 Erro sintatico encontrado\\
--------\\
./exemplos/nao\_aceitos/programanaoaceito8.fun\\
Trabalho de Introdução a Compiladores.\\
Lendo o arquivo : ./exemplos/nao\_aceitos/programanaoaceito8.fun\\
Encountered " <AND> "and "" at line 5, column 51.\\
Was expecting one of:\\
    "+" ...\\
    "-" ...\\
    <NOT> ...\\
    "(" ...\\
    <float\_constant> ...\\
    <integer\_constant> ...\\
    <boolean\_constant> ...\\
    <char\_constant> ...
    <string\_constant> ...\\
    "null" ...\\
    <IDENT> ...\\
\\
O programa foi corretamente analisado\\
1 Erro sintatico encontrado\\
--------\\
./exemplos/nao\_aceitos/programanaoaceito9.fun\\
Trabalho de Introdução a Compiladores.\\
Lendo o arquivo : ./exemplos/nao\_aceitos/programanaoaceito9.fun\\
Encountered " ")" ") "" at line 5, column 35.\\
Was expecting one of:\\
    "+" ...\\
    "-" ...\\
    <NOT> ...\\
    "(" ...\\
    <float\_constant> ...\\
    <integer\_constant> ...\\
    <boolean\_constant> ...\\
    <char\_constant> ...
    <string\_constant> ...\\
    "null" ...\\
    <IDENT> ...\\
\\
O programa foi corretamente analisado\\
1 Erro sintatico encontrado\\
--------\\
./exemplos/nao\_aceitos/programanaoaceito10.fun\\
Trabalho de Introdução a Compiladores.\\
Lendo o arquivo : ./exemplos/nao\_aceitos/programanaoaceito10.fun\\
Encountered " <NOT> "not "" at line 5, column 33.\\
Was expecting one of:\\
    "+" ...\\
    "-" ...\\
    "*" ...\\
    "/" ...\\
    <AND> ...\\
    <OR> ...\\
    <XOR> ...\\
    ">" ...\\
    "<" ...\\
    "==" ...\\
    "<=" ...\\
    ">=" ...
    <NEQ> ...\\
    "%" ...\\
    ";" ...\\
\\
O programa foi corretamente analisado\\
1 Erro sintatico encontrado\\
--------\\
./exemplos/nao\_aceitos/programanaoaceito11.fun\\
Trabalho de Introdução a Compiladores.\\
Lendo o arquivo : ./exemplos/nao\_aceitos/programanaoaceito11.fun
Encountered " "=" "= "" at line 3, column 33.\\
Was expecting:\\
    <IDENT> ...\\
\\
Encountered " "\}" "\} "" at line 5, column 9.\\
Was expecting one of:\\
    "break" ...\\
    "print" ...\\
    "read" ...\\
    "return" ...\\
    "super" ...\\
    "if" ...\\
    "for" ...\\
    "\{" ...
    ";" ...\\
    <IDENT> ...\\
    <IDENT> ...\\
\\
O programa foi corretamente analisado\\
2 Erro sintatico encontrado\\
--------\\
./exemplos/nao\_aceitos/programanaoaceito12.fun\\
Trabalho de Introdução a Compiladores.\\
Lendo o arquivo : ./exemplos/nao\_aceitos/programanaoaceito12.fun\\
Encountered " <string\_constant> "/"aaaa/" "" at line 3, column 44.\\
Was expecting one of:\\
    "=" ...
    ")" ...\\
    "]" ...\\
    "," ...\\
\\
Encountered " "\}" "\} "" at line 6, column 1.
Was expecting one of:\\
    <EOF>\\
    "class" ...\\
\\
O programa foi corretamente analisado\\
2 Erro sintatico encontrado\\
--------\\
./exemplos/nao\_aceitos/programanaoaceito13.fun\\
Trabalho de Introdução a Compiladores.\\
Lendo o arquivo : ./exemplos/nao\_aceitos/programanaoaceito13.fun\\
Encountered " <string\_constant> "/"aaaa/" "" at line 3, column 48.\\
Was expecting one of:\\
    "=" ...
    ")" ...\\
    "]" ...\\
    "," ...\\
\\
Encountered " "int" "int "" at line 5, column 17.
Was expecting one of:\\
    "]" ...\\
    <IDENT> ...\\
\\
O programa foi corretamente analisado\\
2 Erro sintatico encontrado\\
--------\\
./exemplos/nao\_aceitos/programanaoaceito14.fun\\
Trabalho de Introdução a Compiladores.\\
Lendo o arquivo : ./exemplos/nao\_aceitos/programanaoaceito14.fun\\
Encountered " "else" "else "" at line 6, column 9.\\
Was expecting one of:\\
    "break" ...\\
    "print" ...\\
    "read" ...\\
    "return" ...\\
    "super" ...\\
    "if" ...\\
    "for" ...\\
    "\{" ...
    ";" ...\\
    <IDENT> ...\\
    <IDENT> ...\\
\\
Encountered " "==" "== "" at line 7, column 18.\\
Was expecting one of:\\
    "+" ...\\
    "-" ...\\
    <NOT> ...\\
    "(" ...\\
    <float\_constant> ...\\
    <integer\_constant> ...\\
    <boolean\_constant> ...\\
    <char\_constant> ...
    <string\_constant> ...\\
    "null" ...\\
    <IDENT> ...\\
\\
Encountered " <IDENT> "b "" at line 9, column 19.\\
Was expecting one of:\\
    "+" ...\\
    "-" ...\\
    "*" ...\\
    "/" ...\\
    <AND> ...\\
    <OR> ...\\
    <XOR> ...\\
    ">" ...\\
    "<" ...\\
    "==" ...\\
    "<=" ...\\
    ">=" ...\\
    <NEQ> ...\\
    "%" ...
    ")" ...\\
    "[" ...\\
    "." ...\\
\\
Encountered " "+" "+ "" at line 9, column 21.\\
Was expecting one of:
    "=" ...\\
    "[" ...\\
    "." ...\\
\\
O programa foi corretamente analisado\\
4 Erro sintatico encontrado\\
--------\\
./exemplos/nao\_aceitos/programanaoaceito15.fun\\
Trabalho de Introdução a Compiladores.\\
Lendo o arquivo : ./exemplos/nao\_aceitos/programanaoaceito15.fun
Encountered " "extends" "extends "" at line 1, column 7.\\
Was expecting:\\
    <IDENT> ...\\
\\
Encountered " "(" "( "" at line 3, column 16.\\
Was expecting one of:
    "]" ...\\
    <IDENT> ...\\
    <IDENT> ...\\
\\
O programa foi corretamente analisado\\
2 Erro sintatico encontrado\\
--------\\
./exemplos/nao\_aceitos/programanaoaceito16.fun\\
Trabalho de Introdução a Compiladores.\\
Lendo o arquivo : ./exemplos/nao\_aceitos/programanaoaceito16.fun
Encountered " "boolean" "boolean "" at line 3, column 28.\\
Was expecting:\\
    "(" ...\\
\\
Encountered " ";" "; "" at line 3, column 35.
Was expecting one of:\\
    "]" ...\\
    <IDENT> ...\\
\\
O programa foi corretamente analisado\\
2 Erro sintatico encontrado\\
--------\\
./exemplos/nao\_aceitos/programanaoaceito17.fun\\
Trabalho de Introdução a Compiladores.\\
Lendo o arquivo : ./exemplos/nao\_aceitos/programanaoaceito17.fun\\
Encountered " "int" "int "" at line 4, column 13.\\
Was expecting one of:
    "]" ...\\
    <IDENT> ...\\
    <IDENT> ...\\
\\
O programa foi corretamente analisado\\
1 Erro sintatico encontrado\\
--------\\
./exemplos/nao\_aceitos/programanaoaceito18.fun\\
Trabalho de Introdução a Compiladores.\\
Lendo o arquivo : ./exemplos/nao\_aceitos/programanaoaceito18.fun\\
Encountered " "=" "= "" at line 5, column 16.\\
Was expecting one of:
    "]" ...\\
    <IDENT> ...\\
    <IDENT> ...\\
\\
O programa foi corretamente analisado\\
1 Erro sintatico encontrado\\
--------\\
./exemplos/nao\_aceitos/programanaoaceito19.fun\\
Trabalho de Introdução a Compiladores.\\
Lendo o arquivo : ./exemplos/nao\_aceitos/programanaoaceito19.fun\\
Encountered " "if" "if "" at line 4, column 9.\\
Was expecting one of:\\
    "float" ...\\
    "boolean" ...\\
    "char" ...\\
    "string" ...\\
    "int" ...\\
    "constructor" ...\\
    "\}" ...\\
    <IDENT> ...\\
    "int" ...\\
    "string" ...\\
    "float" ...
    "char" ...\\
    "boolean" ...\\
    <IDENT> ...\\
\\
O programa foi corretamente analisado\\
1 Erro sintatico encontrado\\
--------\\
./exemplos/nao\_aceitos/programanaoaceito20.fun\\
Trabalho de Introdução a Compiladores.\\
Lendo o arquivo : ./exemplos/nao\_aceitos/programanaoaceito20.fun\\
Encountered " <NOT> "not "" at line 5, column 32.\\
Was expecting one of:\\
    "+" ...\\
    "-" ...\\
    "*" ...\\
    "/" ...\\
    <AND> ...\\
    <OR> ...\\
    <XOR> ...\\
    ">" ...\\
    "<" ...\\
    "==" ...\\
    "<=" ...\\
    ">=" ...
    <NEQ> ...\\
    "%" ...\\
    ")" ...\\
\\
Encountered " "(" "( "" at line 6, column 27.\\
Was expecting one of:\\
    "=" ...
    "[" ...\\
    ";" ...\\
    "," ...\\
\\
Encountered " ")" ") "" at line 6, column 34.\\
Was expecting one of:
    "=" ...\\
    "[" ...\\
    "." ...\\
\\
EOF found prematurely.\\
EOF found prematurely.\\
EOF found prematurely.\\
EOF found prematurely\\
EOF found prematurely.\\
EOF found prematurely.\\
EOF found prematurely.\\
O programa foi corretamente analisado\\
1 Erro sintatico encontrado\\

\newpage
\section{Conclusão}

Com esse trabalhos conseguimos compreender melhor o funcionamento do tratamento de erros de um analisador sintático assim como o entendimento do uso da ferramenta Javacc para criação de um compilador.

Com o presente concluímos a terceira etapa da série de trabalhos da disciplina com o intuito final de construir todas as etapas de um compilador, gerando assim uma linguagem de programação que possa ser compilada utilizando o material gerado nestes trabalhos.

\bookmarksetup{startatroot}
\end{document}
