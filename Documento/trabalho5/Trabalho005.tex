% abtex2-modelo-artigo.tex, v-1.9.2 laurocesar
% Copyright 2012-2014 by abnTeX2 group at http://abntex2.googlecode.com/ 
%

% ------------------------------------------------------------------------
% ------------------------------------------------------------------------
% abnTeX2: Modelo de Artigo Acadêmico em conformidade com
% ABNT NBR 6022:2003: Informação e documentação - Artigo em publicação 
% periódica científica impressa - Apresentação
% ------------------------------------------------------------------------
% ------------------------------------------------------------------------

\documentclass[
	% -- opções da classe memoir --
	article,			% indica que é um artigo acadêmico
	11pt,				% tamanho da fonte
	oneside,			% para impressão apenas no verso. Oposto a twoside
	a4paper,			% tamanho do papel. 
	% -- opções da classe abntex2 --
	%chapter=TITLE,		% títulos de capítulos convertidos em letras maiúsculas
	%section=TITLE,		% títulos de seções convertidos em letras maiúsculas
	%subsection=TITLE,	% títulos de subseções convertidos em letras maiúsculas
	%subsubsection=TITLE % títulos de subsubseções convertidos em letras maiúsculas
	% -- opções do pacote babel --
	portuguese,			% idioma adicional para hifenização
	brazil,				% o último idioma é o principal do documento
	sumario=tradicional
	]{abntex2}

% ---
% PACOTES
% ---

% ---
% Pacotes fundamentais 
% ---
\usepackage{lmodern}			% Usa a fonte Latin Modern
\usepackage[T1]{fontenc}		% Selecao de codigos de fonte.
\usepackage[utf8]{inputenc}		% Codificacao do documento (conversão automática dos acentos)
\usepackage{indentfirst}		% Indenta o primeiro parágrafo de cada seção.
\usepackage{nomencl} 			% Lista de simbolos
\usepackage{color}				% Controle das cores
\usepackage{graphicx}			% Inclusão de gráficos
\usepackage{microtype} 			% para melhorias de justificação
\usepackage{array}
\usepackage{tabularx}
\usepackage{graphicx}
\usepackage{subcaption}
\usepackage{float}
\usepackage{hyperref}
\usepackage{multirow}
\usepackage{longtable}
\usepackage{xcolor}
% Definindo novas cores
\definecolor{verde}{rgb}{0.25,0.5,0.35}
\definecolor{jpurple}{rgb}{0.5,0,0.35}
\usepackage{dpfloat, booktabs}
\usepackage{listings}
\lstset{
  language=Java,
  basicstyle=\ttfamily\small, 
  keywordstyle=\color{jpurple}\bfseries,
  stringstyle=\color{red},
  commentstyle=\color{verde},
  morecomment=[s][\color{blue}]{/**}{*/},
  extendedchars=true, 
  showspaces=false, 
  showstringspaces=false, 
  numbers=left,
  numberstyle=\tiny,
  breaklines=true, 
  backgroundcolor=\color{cyan!10}, 
  breakautoindent=true, 
  captionpos=b,
  xleftmargin=0pt,
  tabsize=4
}

% ---
\graphicspath{ {images/} }		
% ---
% Pacotes adicionais, usados apenas no âmbito do Modelo Canônico do abnteX2
% ---
\usepackage{lipsum}				% para geração de dummy text
% ---
		
% ---
% Pacotes de citações
% ---
\usepackage[brazilian,hyperpageref]{backref}	 % Paginas com as citações na bibl
\usepackage[alf]{abntex2cite}	% Citações padrão ABNT
% ---

% ---
% Configurações do pacote backref
% Usado sem a opção hyperpageref de backref
\renewcommand{\backrefpagesname}{Citado na(s) página(s):~}
% Texto padrão antes do número das páginas
\renewcommand{\backref}{}
% Define os textos da citação
\renewcommand*{\backrefalt}[4]{
	\ifcase #1 %
		Nenhuma citação no texto.%
	\or
		Citado na página #2.%
	\else
		Citado #1 vezes nas páginas #2.%
	\fi}%
% ---

% ---
% Informações de dados para CAPA e FOLHA DE ROSTO
% ---
\titulo{Introdução a Compiladores - Trabalho 5 }
\autor{Daniel Yoshizawa (13101269)\\Guilherme Nunes (13103611)\\Larissa Taw (14209793)\\Mayse Espíndola (11204360) }
\local{Florianópolis, SC,  Brasil}
\data{20 de junho de 2018}
% ---

% ---
% Configurações de aparência do PDF final

% alterando o aspecto da cor azul
\definecolor{blue}{RGB}{41,5,195}

% informações do PDF
\makeatletter
\hypersetup{
     	%pagebackref=true,
		pdftitle={\@title}, 
		pdfauthor={\@author},
    	pdfsubject={Modelo de artigo científico com abnTeX2},
	    pdfcreator={LaTeX with abnTeX2},
		pdfkeywords={abnt}{latex}{abntex}{abntex2}{atigo científico}, 
		colorlinks=true,       		% false: boxed links; true: colored links
    	linkcolor=blue,          	% color of internal links
    	citecolor=blue,        		% color of links to bibliography
    	filecolor=magenta,      		% color of file links
		urlcolor=blue,
		bookmarksdepth=4
}
\makeatother
% --- 

% ---
% compila o indice
% ---
\makeindex
% ---

% ---
% Altera as margens padrões
% ---
\setlrmarginsandblock{3cm}{3cm}{*}
\setulmarginsandblock{3cm}{3cm}{*}
\checkandfixthelayout
% ---

% --- 
% Espaçamentos entre linhas e parágrafos 
% --- 

% O tamanho do parágrafo é dado por:
\setlength{\parindent}{1.3cm}

\setlength{\parskip}{0.2cm}

\SingleSpacing

\begin{document}

\frenchspacing 

\maketitle

\begin{resumoumacoluna}

Finalização da analise semântica e da serie de trabalhos para a disciplina de introdução à compiladores, neste trabalho foram feitas as implementações da analise semântica, conforme capítulos 10 e 11 do livro base, de declarações de variáveis,  métodos e construtores das classes, também na Parte B foram adicionadas outras funcionalidades como a verificação da definição circular de classes e a verificação de tipos, abrangendo os tipos adicionados a linguagem (\textbf{Float}, \textbf{Char} e \textbf{Boolean}).

 \vspace{\onelineskip}
 
 \noindent
 \textbf{Palavras-chaves}: Introdução a Compiladores, Javacc, Fun, Analisador Semântico
\end{resumoumacoluna}

\newpage
\tableofcontents*
\newpage
\textual
\section{Descrição das atividade}

Nesta etapa do trabalho foi finalizada a implementação do analisador semântico, sendo este dividido em duas partes. Para a \textit{Parte A} foi desenvolvida a análise semântica da herança entre classes, declaração de variáveis, método e construtores de classes, a \textit{Parte B} é um pouco mais extensa e de maior complexidade, aqui foi desenvolvida a verificação da definição circular de classes e verificação de tipos, onde foram introduzidos os tipos \textit{Float}, \textit{Char} e \textit{Boolean}, também foram adicionados os operadores lógicos \textit{not}, \textit{and}, \textit{or} e \textit{xor}, sendo estes tratados como nós relacionais do tipo \textit{RelationalNode}, para isso a analise sintática de expressão foi alterada, assim o código resultante é

\begin{lstlisting}

ExpreNode expression(RecoverySet g) throws ParseEOFException :
{
ExpreNode e1 = null, e2 = null;
Token t = null;

}
{
try {
    e1 = numexpr()
    [
      ( t = <LT> | t = <GT> | t = <LE> | t = <GE> | t = <EQ> | t = <NEQ> | t = <AND> | t = <OR> | t = <XOR> | t = <NOT> )
      e2 = numexpr()
        { e1 = new RelationalNode(t, e1, e2); }
    ]
    { return e1; }
}
catch (ParseException e)
{
   consumeUntil(g, e, "expression");
   return new RelationalNode(t, e1, e2);
}
}
\end{lstlisting}

Foi criada a classe \textit{TypeCheck} para analise de tipos, onde foram adicionados os novos tipos da linguagem, como foram adicionados em vários pontos distintos da implementação apenas alguns serão destacados. Como exemplo a verificação do tipo de retorno em uma operação envolvendo nós relacionais

\begin{lstlisting}
        if ((t1.ty == FLOAT_TYPE) && (t2.ty == FLOAT_TYPE)) {
            return new type(FLOAT_TYPE, 0);
        }

        if ((t1.ty == BOOLEAN_TYPE) && (t2.ty == BOOLEAN_TYPE)) {
            return new type(BOOLEAN_TYPE, 0);
        }

        if ((t1.ty == CHAR_TYPE) && (t2.ty == CHAR_TYPE)) {
            return new type(CHAR_TYPE, 0);
        }
\end{lstlisting}

Assim como a validação de seus tipos constantes, no código a seguir é mostrada a validação de um nó do tipo float, para isso é tentado converter uma \textbf{string} em \textbf{float} utilizando o método \textit{Float.parseFloat(string)} da api do java, se a conversão não for possível uma exceção é lançada avisando que este token se trata de um \textit{float} invalido.

\begin{lstlisting}
    public type TypeCheckFloatConstNode(FloatConstNode x) throws SemanticException
    {
        float k;

        if (x == null) {
            return null;
        }

        try {
            k = Float.parseFloat(x.position.image);
        } catch (NumberFormatException e) {
            throw new SemanticException(x.position, "Invalid float constant");
        }

        return new type(FLOAT_TYPE, 0);
    }
\end{lstlisting}

A parte a seguir de código se refere ao segundo item da \textit{Parte B} do trabalho, onde é verificado o tipo que deve ser retornado caso duas expressões sejam do tipo \textbf{Int} ou \textbf{String}, assim para 2 tipos inteiros é retornado tipo inteiro e para quaisquer outras combinações destes tipos é retornado \textit{string}, caso não atenda a estes critérios uma exceção é lançada.

\begin{lstlisting}
public type TypeCheckAddNode(AddNode x) throws SemanticException {
        type t1;
        type t2;
        int op;
        int i; // Inteiro
        int j; // String


        if (x == null) {
            return null;
        }

        op = x.position.kind;
        t1 = TypeCheckExpreNode(x.expr1);
        t2 = TypeCheckExpreNode(x.expr2);

        if ((t1.dim > 0) || (t2.dim > 0)) {
            throw new SemanticException(x.position,
                "Can not use " + x.position.image + " for arrays");
        }

        i = j = 0;

        if (t1.ty == INT_TYPE) {
            i++;
        } else if (t1.ty == STRING_TYPE) {
            j++;
        }

        if (t2.ty == INT_TYPE) {
            i++;
        } else if (t2.ty == STRING_TYPE) {
            j++;
        }

        if (i == 2) {
            return new type(INT_TYPE, 0);
        }

        if ((op == FunConstants.PLUS) && ((i + j) == 2)) {
            return new type(STRING_TYPE, 0);
        }

        throw new SemanticException(x.position,
            "Invalid types for " + x.position.image);
    }
\end{lstlisting}

\section{Dificuldades encontradas}

Alguns itens não foram resolvidos como a inicialização das variáveis, apesar de diversas tentativas não foi possível implementar essa funcionalidade em nossa linguagem, também o tratamento de atributo opcional não foi possível concluir até a entrega deste trabalho.

\newpage
\section{Conclusão}

Com esse trabalhos concluímos a série de trabalhos da disciplina tendo uma visão geral do funcionamento de um compilador, com esta etapa podemos entender melhor o funcionamento do analisador semântico e como as regras são geradas e validadas de acordo com a linguagem.

\bookmarksetup{startatroot}
\end{document}
